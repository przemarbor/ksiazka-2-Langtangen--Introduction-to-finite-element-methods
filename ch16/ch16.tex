\documentclass[../main.tex]{subfiles}
\begin{document}
	
		\chapter{Implementation}
\label{chap:chap_16}
%\pagenumbering{arabic}

\noindent It is tempting to create a program with symbolic calculations to perform all the steps in the computational machinery, both for automating the work and for documenting the complete algorithms. As we have seen, there are quite many details involved with finite element computations and incorporation of boundary conditions. An implementation will also act as a structured summary of all these details.

	\section[Implementation]{Global basis functions}
		\label{sec:sec_16_1}
		\noindent We first consider implementations when $\psi_{i}$ are global functions are hence different from zero on most of $\Omega=[0, L]$ so all integrals need integration over the entire domain. Since the expressions for the entries in the linear system depend on the differential equation problem being solved, the user must supply the necessary formulas via Python functions. The implementations here attempt to perform symbolic calculations, but fall back on numerical computations if the symbolic ones fail.
		
		The user must prepare a function \mycode{integrand\_lhs(psi, i, j)} for returning the integrand of the integral that contributes to matrix entry $(i, j)$. The \mycode{psi} variable is a Python dictionary holding the basis functions and their derivatives in symbolic form. More precisely, \mycode{psi[q]} is a list of
		$$
		\left\{\frac{d^{q} \psi_{0}}{d x^{q}}, \ldots, \frac{d^{q} \psi_{N}}{d x^{q}}\right\} .
		$$
		Similarly, \mycode{integrand\_rhs(psi, i)} returns the integrand for entry number $i$ in the right-hand side vector.
		
		Since we also have contributions to the right-hand side vector, and potentially also the matrix, from boundary terms without any integral, we introduce two additional functions, \mycode{boundary\_lhs(psi, i, j)} and \mycode{boundary\_rhs(psi, i)} for returning terms in the variational formulation that are not to be integrated over the domain $\Omega$. Examples shown later will explain in more detail how these user-supplied function may look like.
		
		The linear system can be computed and solved symbolically by the following function:
		\begin{lstlisting}[numbers=none]
			import sympy as sp
			def solve(integrand_lhs, integrand_rhs, psi, Omega,
					boundary_lhs=None, boundary_rhs=None):
				N = len(psi[0]) - 1
				A = sp.zeros((N+1, N+1))
				b = sp.zeros((N+1, 1))
				x = sp.Symbol('x')
				for i in range(N+1):
					for j in range(i, N+1):
						integrand = integrand_lhs(psi, i, j)
						I = sp.integrate(integrand, (x, Omega[0], Omega[1]))
						if boundary_lhs is not None:
							I += boundary_lhs(psi, i, j)
						A[i,j] = A[j,i] = I # assume symmetry
					integrand = integrand_rhs(psi, i)
					I = sp.integrate(integrand, (x, Omega[0], Omega[1]))
					if boundary_rhs is not None:
						I += boundary_rhs(psi, i)
					b[i,0] = I
				c = A.LUsolve(b)
				u = sum(c[i,0]*psi[0][i] for i in range(len(psi[0])))
				return u
		\end{lstlisting}\smallbreak
		
		Not surprisingly, symbolic solution of differential equations, discretized by a Galerkin or least squares method with global basis functions, is of limited interest beyond the simplest problems, because symbolic integration might be very time consuming or impossible, not only in \mycode{sympy} but also in \href{https://www.wolframalpha.com/}{WolframAlpha} (which applies the perhaps most powerful symbolic integration software available today: Mathematica). Numerical integration as an option is therefore desirable.\smallbreak
		The extended \mycode{solve} function below tries to combine symbolic and numerical integration. The latter can be enforced by the user, or it can be invoked after a non-successful symbolic integration (being detected by an \mycode{Integral} object as the result of the integration in \mycode{sympy}). Note that for a numerical integration, symbolic expressions must be converted to Python functions (using \mycode{lambdify}), and the expressions cannot contain other symbols than \mycode{x}. The real \mycode{solve} routine in the \textbf{\texttt{\href{https://github.com/hplgit/INF5620/blob/master/src/fem/varform1D.py}{varform1D.py}}} file has error checking and meaningful error messages in such cases. The \mycode{solve} code below is a condensed version of the real one, with the purpose of showing how to automate the Galerkin or least squares method for solving differential equations in $1 \mathrm{D}$ with global basis functions:
		\begin{lstlisting}[numbers=none]
			def solve(integrand_lhs, integrand_rhs, psi, Omega,
				boundary_lhs=None, boundary_rhs=None, symbolic=True):
			N = len(psi[0]) - 1
			A = sp.zeros((N+1, N+1))
			b = sp.zeros((N+1, 1))
			x = sp.Symbol('x')
			for i in range(N+1):
				for j in range(i, N+1):
					integrand = integrand_lhs(psi, i, j)
					if symbolic:
						I = sp.integrate(integrand, (x, Omega[0], Omega[1]))
						if isinstance(I, sp.Integral):
							symbolic = False # force num.int. hereafter
					if not symbolic:
						integrand = sp.lambdify([x], integrand)
						I = sp.mpmath.quad(integrand, [Omega[0], Omega[1]])
					if boundary_lhs is not None:
						I += boundary_lhs(psi, i, j)
					A[i,j] = A[j,i] = I
				integrand = integrand_rhs(psi, i)
				if symbolic:
					I = sp.integrate(integrand, (x, Omega[0], Omega[1]))
					if isinstance(I, sp.Integral):
						symbolic = False
				if not symbolic:
					integrand = sp.lambdify([x], integrand)
					I = sp.mpmath.quad(integrand, [Omega[0], Omega[1]])
				if boundary_rhs is not None:
					I += boundary_rhs(psi, i)
				b[i,0] = I
			c = A.LUsolve(b)
			u = sum(c[i,0]*psi[0][i] for i in range(len(psi[0])))
			return u
		\end{lstlisting}
	\section[Example: constant right-hand side]{Example: constant right-hand side}
		\label{sec:sec_16_2}
		\noindent To demonstrate the code above, we address
		$$
		-u^{\prime \prime}(x)=b, \quad x \in \Omega=[0,1], \quad u(0)=1, u(1)=0,
		$$
		with $b$ as a (symbolic) constant. A possible basis for the space $V$ is $\psi_{i}(x)=$ $x^{i+1}(1-x), i \in \mathcal{I}_{s}$. Note that $\psi_{i}(0)=\psi_{i}(1)=0$ as required by the Dirichlet conditions. We need a $B(x)$ function to take care of the known boundary values of $u$. Any function $B(x)=1-x^{p}, p \in \mathbb{R}$, is a candidate, and one arbitrary choice from this family is $B(x)=1-x^{3}$. The unknown function is then written as
		$$
		u(x)=B(x)+\sum_{j \in \mathcal{I}_{s}} c_{j} \psi_{j}(x)
		$$\smallbreak
		Let us use the Galerkin method to derive the variational formulation. Multiplying the differential equation by $v$ and integrate by parts yield
		$$
		\int_{0}^{1} u^{\prime} v^{\prime} \mathrm{d} x=\int_{0}^{1} f v \mathrm{~d} x \quad \forall v \in V,
		$$
		and with $u=B+\sum_{j} c_{j} \psi_{j}$ we get the linear system
		
		\begin{equation}
			\label{eqa196}
			\sum_{j \in \mathcal{I}_{s}}\left(\int_{0}^{1} \psi_{i}^{\prime} \psi_{j}^{\prime} \mathrm{d} x\right) c_{j}=\int_{0}^{1}\left(f-B^{\prime}\right) \psi_{i} \mathrm{~d} x, \quad i \in \mathcal{I}_{s}
		\end{equation}
		
		The application can be coded as follows in \mycode{sympy}:
		
		\begin{lstlisting}[numbers=none]
			x, b = sp.symbols('x b')
			f = b
			B = 1 - x**3
			dBdx = sp.diff(B, x)
			
			# Compute basis functions and their derivatives
			N = 3
			psi = {0: [x**(i+1)*(1-x) for i in range(N+1)]}
			psi[1] = [sp.diff(psi_i, x) for psi_i in psi[0]]
			
			def integrand_lhs(psi, i, j):
				return psi[1][i]*psi[1][j]
				
			def integrand_rhs(psi, i):
				return f*psi[0][i] - dBdx*psi[1][i]
				
			Omega = [0, 1]
			
			u_bar = solve(integrand_lhs, integrand_rhs, psi, Omega,
			verbose=True, symbolic=True)
			
			u = B + u_bar
			print 'solution u:', sp.simplify(sp.expand(u))
		\end{lstlisting}
	
	\noindent The printout of $\mathrm{u}$ reads \textbf{\texttt{-b*x**2/2 + b*x/2 - x + 1}}. Note that expanding $\mathrm{u}$ and then simplifying is in the present case necessary to get a compact, final expression with sympy. A non-expanded u might be preferable in other cases this depends on the problem in question.\smallbreak
	The exact solution $u_{e}(x)$ can be derived by some \mycode{sympy} code that closely follows the examples in Section \ref{sec:sec_11_2}. The idea is to integrate $-u^{\prime \prime}=b$ twice and determine the integration constants from the boundary conditions:

		\begin{lstlisting}[numbers=none]
			C1, C2 = sp.symbols('C1 C2') # integration constants
			f1 = sp.integrate(f, x) + C1
			f2 = sp.integrate(f1, x) + C2
			# Find C1 and C2 from the boundary conditions u(0)=0, u(1)=1
			s = sp.solve([u_e.subs(x,0) - 1, u_e.subs(x,1) - 0], [C1, C2])
			# Form the exact solution
			u_e = -f2 + s[C1]*x + s[C2]
			print 'analytical solution:', u_e
			print 'error:', sp.simplify(sp.expand(u - u_e))
		\end{lstlisting}
	
	
	\noindent The last line prints 0 , which is not surprising when $u_{\mathrm{e}}(x)$ is a parabola and our approximate $u$ contains polynomials up to degree 4. It suffices to have $N=1$, i.e., polynomials of degree 2 , to recover the exact solution.
	
	We can play around with the code and test that with $f \sim x^{p}$, the solution is a polynomial of degree $p+2$, and $N=p+1$ guarantees that the approximate solution is exact.
	
	Although the symbolic code is capable of integrating many choices of $f(x)$, the symbolic expressions for $u$ quickly become lengthy and non-informative, so numerical integration in the code, and hence numerical answers, have the greatest application potential.\bigbreak
	
	\section[Finite elements]{Finite elements}
		\label{sec:sec_16_3}
		 \noindent Implementation of the finite element algorithms for differential equations follows closely the algorithm for approximation of functions. The new additional ingredients are
		 \begin{enumerate}
		 	\item other types of integrands (as implied by the variational formulation)
		  	\item additional boundary terms in the variational formulation for Neumann boundary conditions
		 	\item modification of element matrices and vectors due to Dirichlet boundary conditions
		\end{enumerate}
		 
		 Point 1 and 2 can be taken care of by letting the user supply functions defining the integrands and boundary terms on the left- and right-hand side of the equation system:
		 
		\begin{lstlisting}[numbers=none]
			integrand_lhs(phi, r, s, x)
			boundary_lhs(phi, r, s, x)
			integrand_rhs(phi, r, x)
			boundary_rhs(phi, r, x)
		\end{lstlisting}
	 
	 	\noindent Here, \mycode{phi} is a dictionary where \mycode{phi[q]} holds a list of the derivatives of order $q$ of the basis functions at the an evaluation point; \mycode{r} and \mycode{s} are indices for the corresponding entries in the element matrix and vector, and \mycode{x} is the global coordinate value corresponding to the current evaluation point.
	 	
	 	Given a mesh represented by \mycode{vertices}, \mycode{cells}, and \mycode{dof\_map} as explained before, we can write a pseudo Python code to list all the steps in the computational algorithm for finite element solution of a differential equation.
	 	
	 	\begin{lstlisting}[numbers=none]
			<Declare global matrix and rhs: A, b>
			
			for e in range(len(cells)):
			
				# Compute element matrix and vector
				n = len(dof_map[e]) # no of dofs in this element
				h = vertices[cells[e][1]] - vertices[cells[e][1]]
				<Declare element matrix and vector: A_e, b_e>
			
				# Integrate over the reference cell
				points, weights = <numerical integration rule>
				for X, w in zip(points, weights):
					phi = <basis functions and derivatives at X>
					detJ = h/2
					x = <affine mapping from X>
					for r in range(n):
						for s in range(n):
							A_e[r,s] += integrand_lhs(phi, r, s, x)*detJ*w
						b_e[r] += integrand_rhs(phi, r, x)*detJ*w
			
				# Add boundary terms
				for r in range(n):
					for s in range(n):
						A_e[r,s] += boundary_lhs(phi, r, s, x)*detJ*w
					b_e[r] += boundary_rhs(phi, r, x)*detJ*w
			
				# Incorporate essential boundary conditions
				for r in range(n):
					global_dof = dof_map[e][r]
					if global_dof in essbc_dofs:
						# dof r is subject to an essential condition
					value = essbc_docs[global_dof]
					# Symmetric modification
					b_e -= value*A_e[:,r]
					A_e[r,:] = 0
					A_e[:,r] = 0
					A_e[r,r] = 1
					b_e[r] = value
			
				# Assemble
				for r in range(n):
					for s in range(n):
						A[dof_map[e][r], dof_map[e][r]] += A_e[r,s]
					b[dof_map[e][r] += b_e[r]
			<solve linear system>
		\end{lstlisting}
\clearpage
\end{document} 
