\documentclass[../main.tex]{subfiles}
\begin{document}
	\chapter{Summary}
\label{chap:chap_18}
%\pagenumbering{arabic}
\begin{itemize}
	\item When approximating $f$ by $u=\sum_{j} c_{j} \varphi_{j}$, the least squares method and the Galerkin/projection method give the same result. The interpolation/collocation method is simpler and yields different (mostly inferior) results.
	\item Fourier series expansion can be viewed as a least squares or Galerkin approximation procedure with sine and cosine functions.
	\item Basis functions should optimally be orthogonal or almost orthogonal, because this gives little round-off errors when solving the linear system, and the coefficient matrix becomes diagonal or sparse.
	\item Finite element basis functions are \emph{piecewise} polynomials, normally with discontinuous derivatives at the cell boundaries. The basis functions overlap very little, leading to stable numeries and sparse matrices.
	\item To use the finite element method for differential equations, we use the Galerkin method or the method of weighted residuals to arrive at a variational form. Technically, the differential equation is multiplied by a test function and integrated over the domain. Second-order derivatives are integrated by parts to allow for typical finite element basis functions that have discontinuous derivatives.
	\item The least squares method is not much used for finite element solution of differential equations of second order, because it then involves second-order derivatives which cause trouble for basis functions with discontinuous derivatives.
	\item We have worked with two common finite element terminologies and associated data structures (both are much used, especially the first one, while the other is more general):
		\begin{enumerate}
			\item[1.] \emph{elements, nodes, and mapping between local and global node numbers}
			\item[2.] \emph{an extended element concept consisting of cell, vertices, degrees of freedom, local basis functions, geometry mapping, and mapping between local and global degrees of freedom}
		\end{enumerate}
	\item The meaning of the word "element" is multi-fold: the geometry of a finite element (also known as a cell), the geometry and its basis functions, or all information listed under point 2 above.
	\item One normally computes integrals in the finite element method element by element (cell by cell), either in a local reference coordinate system or directly in the physical domain.
	\item The advantage of working in the reference coordinate system is that the mathematical expressions for the basis functions depend on the element type only, not the geometry of that element in the physical domain. The disadvantage is that a mapping must be used, and derivatives must be transformed from reference to physical coordinates.
	\item Element contributions to the global linear system are collected in an element matrix and vector, which must be assembled into the global system using the degree of freedom mapping (\textbf{\texttt{dof\_map}}) or the node numbering mapping (\textbf{\texttt{elements}}), depending on which terminology that is used.
	\item Dirichlet conditions, involving prescribed values of $u$ at the boundary, are implemented either via a boundary function that take on the right Dirichlet values, while the basis functions vanish at such boundaries. In the finite element method, one has a general expression for the boundary function, but one can also incorporate Dirichlet conditions in the element matrix and vector or in the global matrix system.
	\item Neumann conditions, involving prescribed values of the derivative (or flux) of $u$, are incorporated in boundary terms arising from integrating terms with second-order derivatives by part. Forgetting to account for the boundary terms implies the condition $\partial u / \partial n=0$ at parts of the boundary where no Dirichlet condition is set.
\end{itemize}
\clearpage
\end{document} 
