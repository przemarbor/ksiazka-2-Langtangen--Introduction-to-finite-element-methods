\documentclass[../main.tex]{subfiles}
\begin{document}


\chapter{Exercises}
	\section*{Exercise 23: Refactor functions into a more general class}
		\label{sec:sec_21_23}
		Section \ref{sec:sec_11_2} displays three functions for computing the analytical solution of some simple model problems. There is quite some repetitive code, suggesting that the functions can benefit from being refactored into a class where the user can define the $f(x), a(x)$, and the boundary conditions in particular methods in subclasses. Demonstrate how the new class can be used to solve the three particular problems in Section \ref{sec:sec_11_2}.
		
		In the method that computes the solution, check that the solution found fulfills the differential equation and the boundary conditions. 
		Filename: \mycode{uxx\_f\_sympy\_class.py}.\bigbreak
	\section*{Exercise 24: Compute the deflection of a cable with sine functions}
		\label{sec:sec_21_24}
		\noindent A hanging cable of length $L$ with significant tension has a downward deflection $w(x)$ governed by
		Solve
		$$
		T w^{\prime \prime}(x)=\ell(x),
		$$
		where $T$ is the tension in the cable and $\ell(x)$ the load per unit length. The cable is fixed at $x=0$ and $x=L$ so the boundary conditions become $T(0)=T(L)=0$. We assume a constant load $\ell(x)=$ const.
		
		The solution is expected to be symmetric around $x=L / 2$. Formulating the problem for $x \in \Omega=[0, L / 2]$ and then scaling it, results in the scaled problem for the dimensionless vertical deflection $u$ :
		$$
		u^{\prime \prime}=1, \quad x \in(0,1), \quad u(0)=0, u^{\prime}(1)=0 .
		$$
		Introduce the function space spanned by $\psi_{i}=\sin ((i+1) \pi x / 2), i=1, \ldots, N$. Use a Galerkin and a least squares method to find the coefficients $c_{j}$ in $u(x)=$ $\sum_{j} c_{j} \psi_{j}$. Find how fast the coefficients decrease in magnitude by looking at $c_{j} / c_{j-1}$. Find the error in the maximum deflection at $x=1$ when only one basis function is used $(N=0)$.
		
		What happens if we choose basis functions $\psi_{i}=\sin ((i+1) \pi x)$ ? These basis functions are appropriate if we do not utilize symmetry and solve the problem on $[0, L]$. A scaled version of this problem reads
		$$
		u^{\prime \prime}=1, \quad x \in(0,1), \quad u(0)=u(1)=0 .
		$$
		Carry out the computations with $N=0$ and demonstrate that the maximum deflection $u(1 / 2)$ is the same in the problem utilizing symmetry and the problem covering the whole cable.
		Filename: \mycode{cable\_sin.pdf}. \bigbreak
	\section*{Exercise 25: Check integration by parts}
		Consider the Galerkin method for the problem involving $u$ in Exercise \hyperref[sec:sec_21_24]{24}. Show that the formulas for $c_{j}$ are independent of whether we perform integration by parts or not. Filename: \mycode{cable\_integr\_by\_parts.pdf}.\bigbreak
	\section*{Exercise 26: Compute the deflection of a cable with 2 P1 elements}
		Solve the problem for $u$ in Exercise \hyperref[sec:sec_21_24]{24} using two P1 linear elements. Filename: \mycode{cable\_2P1.pdf}.\bigbreak
	\section*{Exercise 27: Compute the deflection of a cable with 1 P2 element}
		Solve the problem for $u$ in Exercise \hyperref[sec:sec_21_24]{24} using one P2 element with quadratic basis functions. Filename: \mycode{cable\_1P2.pdf.} \bigbreak
	\section*{Exercise 28: Compute the deflection of a cable with a step load}
		\noindent We consider the deflection of a tension cable as described in Exercise \hyperref[sec:sec_21_24]{24}. Now the load is
		$$
		\ell(x)=\left\{\begin{array}{ll}
			\ell_{1}, & x<L / 2, \\
			\ell_{2}, & x \geq L / 2
		\end{array} \quad x \in[0, L] .\right.
		$$
		This load is not symmetric with respect to the midpoint $x=L / 2$ so the solution loses its symmetry and we must solve the scaled problem
		$$
		u^{\prime \prime}=\left\{\begin{array}{ll}
			1, & x<1 / 2, \\
			0, & x \geq 1 / 2
		\end{array} \quad x \in(0,1), \quad u(0)=0, u(1)=0\right.
		$$
		\begin{enumerate}
			\item[a)] Use $\psi_{i}=\sin ((i+1) \pi x), i=0, \ldots, N$ and the Galerkin method without integration by parts. Derive a formula for $c_{j}$ in the solution expansion $u=$ $\sum_{j} c_{j} \psi_{j}$. Plot how fast the coefficients $c_{j}$ tend to zero (on a log scale).
			\item[b)]Solve the problem with P1 finite elements. Plot the solution for $N_{e}=2,4,8$ elements.
		\end{enumerate}
		Filename: \mycode{cable\_discont\_load.pdf}.\bigbreak
	\section*{Exercise 29: Show equivalence between linear systems}
		\noindent Incorporation of Dirichlet conditions at $x=0$ and $x=L$ in a finite element mesh on $\Omega=[0, L]$ can either be done by introducing an expansion $u(x)=$ $U_{0} \varphi_{0}+U_{N_{n}} \varphi_{N_{n}}+\sum_{j=0}^{N} c_{j} \varphi_{\nu(j)}$, with $N=N_{n}-2$ and considering $u$ values at the inner nodes as unknowns, or one can assemble the matrix system with $u(x)=\sum_{j=0}^{N=N_{n}} c_{j} \varphi_{j}$ and afterwards replace the rows corresponding to known $c_{j}$ values by the boundary conditions. Show that the two approaches are equivalent.\bigbreak
	\section*{Exercise 30: Compute with a non-uniform mesh}
		\noindent Derive the linear system for the problem $-u^{\prime \prime}=2$ on $[0,1]$, with $u(0)=0$ and $u(1)=1$, using $\mathrm{P} 1$ elements and a non-uniform mesh. The vertices have coordinates $x_{0}=0<x_{1}<\cdots<x_{N}=1$, and the length of cell number $e$ is $h_{e}=x_{e+1}-x_{e}$.\smallbreak
		It is of interest to compare the discrete equations for the finite element method in a non-uniform mesh with the corresponding discrete equations arising from a finite difference method. Go through the derivation of the finite difference formula $u^{\prime \prime}\left(x_{i}\right) \approx\left[D_{x} D_{x} u\right]_{i}$ and modify it to find a natural discretization of $u^{\prime \prime}\left(x_{i}\right)$ on a non-uniform mesh. \mycode{Filename: nonuniform\_P1 pdf}.\bigbreak		
	\section*{Problem 31: Solve a 1D finite element problem by hand}
		\noindent The following scaled 1D problem is a very simple, yet relevant, model for convective transport in fluids:
		\begin{equation}
			\label{309}
			u^{\prime}=\epsilon u^{\prime \prime}, \quad u(0)=0, u(1)=1, x \in[0,1]
		\end{equation}
		\begin{enumerate}
			\item[a)] Find the analytical solution to this problem. (Introduce $w=u^{\prime}$, solve the first-order differential equation for $w(x)$, and integrate once more.)
			\item [b)] Derive the variational form of this problem.
			\item[c)] Introduce a finite element mesh with uniform partitioning. Use P1 elements and compute the element matrix and vector for a general element.
			\item[d)] Incorporate the boundary conditions and assemble the element contributions.
			\item[e)] Identify the resulting linear system as a finite difference discretization of the differential equation using
			$$	\left[D_{2 x} u=\epsilon D_{x} D_{x} u\right]_{i} $$
			\item[f)] Compute the numerical solution and plot it together with the exact solution for a mesh with 20 elements and $\epsilon=10,1,0.1,0.01$.
		\end{enumerate}
		\mycode{Filename: convdiff1D\_P1.pdf}.\bigbreak 
	\section*{Exercise 32: Compare finite elements and differences for a radially symmetric Poisson equation}
		\noindent We consider the Poisson problem in a disk with radius $R$ with Dirichlet conditions at the boundary. Given that the solution is radially symmetric and hence dependent only on the radial coordinate $\left(r=\sqrt{x^{2}+y^{2}}\right)$, we can reduce the problem to a 1D Poisson equation
		\begin{equation}
			\label{eqa310}
			-\frac{1}{r} \frac{d}{d r}\left(r \frac{d u}{d r}\right)=f(r), \quad r \in(0, R), u^{\prime}(0)=0, u(R)=U_{R} .
		\end{equation}
		\begin{enumerate}
			\item [a)] Derive a variational form of (\ref{eqa310}) by integrating over the whole disk, or posed equivalently: use a weighting function $2 \pi r v(r)$ and integrate $r$ from 0 to $R$.
			\item[b)] Use a uniform mesh partition with P1 elements and show what the resulting set of equations becomes. Integrate the matrix entries exact by hand, but use a Trapezoidal rule to integrate the $f$ term.
			\item[c)]Explain that an intuitive finite difference method applied to (\ref{eqa310}) gives
		\end{enumerate}
		$$
		\frac{1}{r_{i}} \frac{1}{h^{2}}\left(r_{i+\frac{1}{2}}\left(u_{i+1}-u_{i}\right)-r_{i-\frac{1}{2}}\left(u_{i}-u_{i-1}\right)\right)=f_{i}, \quad i=r h \text {. }
		$$\smallbreak
		For $i=0$ the factor $1 / r_{i}$ seemingly becomes problematic. One must always have $u^{\prime}(0)=0$, because of the radial symmetry, which implies $u_{-1}=u_{1}$, if we allow introduction of a fictitious value $u_{-1}$. Using this $u_{-1}$ in the difference equation for $i=0$ gives
		$$
		\begin{aligned}
			&\frac{1}{r_{0}} \frac{1}{h^{2}}\left(r_{\frac{1}{2}}\left(u_{1}-u_{0}\right)-r_{-\frac{1}{2}}\left(u_{0}-u_{1}\right)\right)= \\
			&\quad \frac{1}{r_{0}} \frac{1}{2 h^{2}}\left(\left(r_{0}+r_{1}\right)\left(u_{1}-u_{0}\right)-\left(r_{-1}+r_{0}\right)\left(u_{0}-u_{1}\right)\right) \approx 2\left(u_{1}-u_{0}\right)
		\end{aligned}
		$$
		if we use $r_{-1}+r_{1} \approx 2 r_{0}$.\smallbreak
		Set up the complete set of equations for the finite difference method and compare to the finite element method in case a Trapezoidal rule is used to integrate the $f$ term in the latter method.
		Filename:\mycode{radial\_Poisson1D\_P1.pdf}. \bigbreak
	\section*{Exercise 33: Compute with variable coefficients and P1 elements by hand}
		\noindent Consider the problem
		\begin{equation}
			\label{311}
			-\frac{d}{d x}\left(a(x) \frac{d u}{d x}\right)+\gamma u=f(x), \quad x \in \Omega=[0, L], \quad u(0)=\alpha, u^{\prime}(L)=\beta .
		\end{equation}
		We choose $a(x)=1+x^{2}$. Then
		\begin{equation}
			\label{312}
			u(x)=\alpha+\beta\left(1+L^{2}\right) \tan ^{-1}(x),
		\end{equation}
		is an exact solution if $f(x)=\gamma u$.\smallbreak
		Derive a variational formulation and compute general expressions for the element matrix and vector in an arbitrary element, using P1 elements and a uniform partitioning of $[0, L]$. The right-hand side integral is challenging and can be computed by a numerical integration rule. The Trapezoidal rule (101) gives particularly simple expressions. Filename: \mycode{atan1D\_P1.pdf}.\bigbreak
	\section*{Exercise 34: Solve a 2D Poisson equation using polynomials and sines}
		\noindent The classical problem of applying a torque to the ends of a rod can be modeled by a Poisson equation defined in the cross section $\Omega$ :
		$$
		-\nabla^{2} u=2, \quad(x, y) \in \Omega,
		$$
		with $u=0$ on $\partial \Omega$. Exactly the same problem arises for the deflection of a membrane with shape $\Omega$ under a constant load.\smallbreak
		For a circular cross section one can readily find an analytical solution. For a rectangular cross section the analytical approach ends up with a sine series. The idea in this exercise is to use a single basis function to obtain an approximate answer.\smallbreak
		We assume for simplicity that the cross section is the unit square: $\Omega=$ $[0,1] \times[0,1]$.
		\begin{enumerate}
			\item[a)] We consider the basis $\psi_{p, q}(x, y)=\sin ((p+1) \pi x) \sin (q \pi y), p, q=0, \ldots, n$. These basis functions fulfill the Dirichlet condition. Use a Galerkin method and $n=0$.
			\item[b)] The basis function involving sine functions are orthogonal. Use this property in the Galerkin method to derive the coefficients $c_{p, q}$ in a formula $u=\sum_{p} \sum_{q} c_{p, q} \psi_{p, q}(x, y) .$
			\item [c)] Another possible basis is $\psi_{i}(x, y)=(x(1-x) y(1-y))^{i+1}, i=0, \ldots, N$. Use the Galerkin method to compute the solution for $N=0$. Which choice of a single basis function is best, $u \sim x(1-x) y(1-y)$ or $u \sim \sin (\pi x) \sin (\pi y)$ ? In order to answer the question, it is necessary to search the web or the literature for an accurate estimate of the maximum $u$ value at $x=y=1 / 2$.
		\end{enumerate}
			Filename: \mycode{torsion\_sin\_xy.pdf}.\bigbreak
	\section*{Exercise 35: Analyze a Crank-Nicolson scheme for the diffusion equation}		
		\noindent Perform the analysis in Section \ref{sec:sec_19_10} for a $1 \mathrm{D}$ diffusion equation $u_{t}=\alpha u_{x x}$ discretized by the Crank-Nicolson scheme in time:
		$$
		\frac{u^{n+1}-u^{n}}{\Delta t}=\alpha \frac{1}{2}\left(\frac{u^{n+1}}{\partial x^{2}} \frac{u^{n}}{\partial x^{2}}\right)
		$$
		or written compactly with finite difference operators,
		$$
		\left[D_{t} u=\alpha D_{x} D_{x} \bar{u}^{t}\right]^{n+\frac{1}{2}}
		$$
		(From a strict mathematical point of view, the $u^{n}$ and $u^{n+1}$ in these equations should be replaced by $u_{e}^{n}$ and $u_{e}^{n+1}$ to indicate that the unknown is the exact solution of the PDE discretized in time, but not yet in space, see Section \ref{sec:sec_19_1}.) Make plots similar to those in Section \ref{sec:sec_19_10}. 
		Filename: \mycode{fe\_diffusion.pdf}.\bigbreak

%\begin{theindex}
%	\item 
%		\subitem 
%		\subitem 
%	\indexspace
%	\item 
%		\subitem 
%		\subitem 
%			\subsubitem 
%			\subsubitem 
%\end{theindex}


\clearpage
\end{document} 
